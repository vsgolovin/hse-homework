\problemset{Контрольная работа №2 (вариант 11)}

% 1
\begin{problem}
    Найти наилучшее приближение $A_1$ ранга $2$ матрицы $A$ по норме
    $||\cdot||_2$ и вычислить\\ $||A - A_1||_2$, где
    \begin{equation*}
        A =
        \begin{bmatrix}
            -8 & -42 & 9 & 40\\
            10 & 48 & -54 & -2\\
            -67 & -12 & 36 & -22
        \end{bmatrix}.
    \end{equation*}
\end{problem}


Сначала выполним сингулярное разложение $A = U \Sigma V^T$. Матрица $V$
состоит из собстенных векторов $A^TA$.
\begin{equation*}
    A^TA = \begin{pmatrix}
        4653 & 1620 & -3024 & 1134\\
        1620 & 4212 & -3402 & -1512\\
        -3024 & -3402 & 4293 & -324\\
        1134 & -1512 & -324 & 2088
    \end{pmatrix}
    \qquad
    V = \frac{1}{11}
    \begin{pmatrix}
        -6 & -6 & 7 & 0\\
        -6 & 6 & 0 & 7\\
        7 & 0 & 6 & 6\\
        0 & -7 & -6 & 6
    \end{pmatrix}
    \qquad
    \lambda = (9801, 4356, 1089, 0)
\end{equation*}
Корни из собственных значений $\lambda$ будут лежать на главной диагонали
матрицы $\Sigma$
\begin{equation*}
    \Sigma = \begin{pmatrix}
        99 & 0 & 0 & 0\\
        0 & 66 & 0 & 0\\
        0 & 0 & 33 & 0
    \end{pmatrix}
\end{equation*}
Далее, воспользовавшись соотношением $AV = U\Sigma$, рассчитаем $U$
\begin{equation*}
    U = \frac{1}{3}
    \begin{pmatrix}
        1 & -2 & -2\\
        -2 & 1 & -2\\
        2 & 2 & -1
    \end{pmatrix}
\end{equation*}
Теперь, воспользовавшись теоремой Эккарта--Янга, найдём $A_1$
\begin{equation*}
    A_1 = U \Sigma_r V^T = U
    \begin{pmatrix}
        99 & 0 & 0 & 0\\
        0 & 66 & 0 & 0\\
        0 & 0 & 0 & 0
    \end{pmatrix}
    V^T =
    \begin{pmatrix}
        6 & -42 & 21 & 28\\
        24 & 48 & -42 & -14\\
        -60 & -12 & 42 & -28
    \end{pmatrix}
\end{equation*}
Погрешность приближения
\begin{equation*}
    ||A - A_1||_2 = ||\Sigma - \Sigma_r||_2 = 33
\end{equation*}


% 2