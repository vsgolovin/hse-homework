\renewcommand{\leq}{\leqslant}
\renewcommand{\geq}{\geqslant}
\newcommand{\eps}{\varepsilon}


\problemset{Контрольная работа №2 (вариант 11)}

% 1
\begin{problem}
    Найти наилучшее приближение $A_1$ ранга $2$ матрицы $A$ по норме
    $||\cdot||_2$ и вычислить\\ $||A - A_1||_2$, где
    \begin{equation*}
        A =
        \begin{pmatrix}
            -8 & -42 & 9 & 40\\
            10 & 48 & -54 & -2\\
            -67 & -12 & 36 & -22
        \end{pmatrix}.
    \end{equation*}
\end{problem}


Сначала выполним сингулярное разложение $A = U \Sigma V^T$. Матрица $V$
состоит из собстенных векторов $A^TA$.
\begin{equation*}
    A^TA = \begin{pmatrix}
        4653 & 1620 & -3024 & 1134\\
        1620 & 4212 & -3402 & -1512\\
        -3024 & -3402 & 4293 & -324\\
        1134 & -1512 & -324 & 2088
    \end{pmatrix}
    \qquad
    V = \frac{1}{11}
    \begin{pmatrix}
        -6 & -6 & 7 & 0\\
        -6 & 6 & 0 & 7\\
        7 & 0 & 6 & 6\\
        0 & -7 & -6 & 6
    \end{pmatrix}
    \qquad
    \lambda = (9801, 4356, 1089, 0)
\end{equation*}
Корни из собственных значений $\lambda$ будут лежать на главной диагонали
матрицы $\Sigma$
\begin{equation*}
    \Sigma = \begin{pmatrix}
        99 & 0 & 0 & 0\\
        0 & 66 & 0 & 0\\
        0 & 0 & 33 & 0
    \end{pmatrix}
\end{equation*}
Далее, воспользовавшись соотношением $AV = U\Sigma$, рассчитаем $U$
\begin{equation*}
    U = \frac{1}{3}
    \begin{pmatrix}
        1 & -2 & -2\\
        -2 & 1 & -2\\
        2 & 2 & -1
    \end{pmatrix}
\end{equation*}
Теперь, воспользовавшись теоремой Эккарта--Янга, найдём $A_1$
\begin{equation*}
    A_1 = U \Sigma_r V^T = U
    \begin{pmatrix}
        99 & 0 & 0 & 0\\
        0 & 66 & 0 & 0\\
        0 & 0 & 0 & 0
    \end{pmatrix}
    V^T =
    \begin{pmatrix}
        6 & -42 & 21 & 28\\
        24 & 48 & -42 & -14\\
        -60 & -12 & 42 & -28
    \end{pmatrix}
\end{equation*}
Погрешность приближения
\begin{equation*}
    ||A - A_1||_2 = ||\Sigma - \Sigma_r||_2 = 33
\end{equation*}


% 2
\newpage
\begin{problem}
    Оценить относительную погрешность приближенного решения $(1, 1)$ системы
    $Ax = b$ по нормам $|\cdot|_1$ и $|\cdot|_2$ с помощью числа обусловленности
    матрицы $A$, где
    \begin{equation*}
        A = \begin{pmatrix}
            4.99 & 0.02\\
            -4.83 & -8.01
        \end{pmatrix}
    ,\qquad
    b = \begin{pmatrix}
            5.0\\
            -12.98
        \end{pmatrix}
    \end{equation*}
\end{problem}


Очевидно, что решение $\hat{x} = (1, 1)$ получается при округлении $A$ и $b$ до
целочисленных значений. Таким образом
\begin{equation*}
    \hat{A} =
    \begin{pmatrix}
        5 & 0\\
        -5 & -8
    \end{pmatrix}
    \qquad
    \eps_A = \hat{A} - A =
    \begin{pmatrix}
        0.01 & -0.02\\
        -0.17 & 0.01
    \end{pmatrix}
    \qquad
    \hat{b} =
    \begin{pmatrix}
        5\\
        -13
    \end{pmatrix}
    \qquad
    \Delta{b} = \hat{b} - b =
    \begin{pmatrix}
        0\\
        -0.02
    \end{pmatrix}
\end{equation*}

Погрешность решения относительно некоторой нормы $||\cdot||$ определяется
выражением
\begin{equation*}
    \delta{x} \leq \frac{\kappa(A)}{1-\kappa(A)\delta{A}}(\delta{A}+\delta{b}),
\end{equation*}
где $\kappa(A) = ||A|| \; ||A^{-1}||$ -- число обусловленности матрицы $A$;
$\delta{A} = ||\eps_A|| / ||A||$, $\delta{b} = ||\Delta{b}|| / ||b||$. Обратная
матрица
\begin{equation*}
    A^{-1} = \frac{1}{|A|}
    \begin{pmatrix}
        -8.01 & -0.02\\
        4.83 & 4.99
    \end{pmatrix}
    = \frac{1}{39.87}
    \begin{pmatrix}
        8.01 & 0.02\\
        -4.83 & -4.99
    \end{pmatrix}
\end{equation*}

Погрешность по норме $||\cdot||_1$:
\begin{equation*}
    \delta{x} \leq
    \frac{9.82\cdot{0.32}}{1-9.82\cdot{0.32}\cdot{(0.18/9.82)}}
    \left(\frac{0.18}{9.82} + \frac{0.02}{17.98}\right)
    =
    \frac{3.16}{1 - 3.16\cdot{0.018}} \left(3.16 + 0.001\right)
    =
    0.065
\end{equation*}

Погрешность по норме $||\cdot||_2$ (наибольшее сингулярное число):
\begin{equation*}
    \delta{x} \leq
    \frac{9.79\cdot{0.25}}{1-9.79\cdot{0.25}\cdot{(0.17/9.78)}}
    \left(\frac{0.17}{9.78} + \frac{0.02}{13.91}\right)
    =
    0.047
\end{equation*}


% 3
\begin{problem}
    Найти приближенно обратную матрицу к матрице $A$ и оценить погрешность
    приближения относительно равномерной нормы $||\cdot||_1$ если элементы
    матрицы $A$ известны с абсолютной погрешностью $0.01$.
    \begin{equation*}
        A \approx \begin{pmatrix}
            2 & -3\\
            8 & -7
        \end{pmatrix}
    \end{equation*}
\end{problem}


Если погрешность элементов матрицы $A$ равна $0.01$, то $||\eps_A||_1 = 0.02$.
Приближенное значение обратной матрицы, число обусловленности и относительное
возмущение:
\begin{equation*}
    A^{-1} = \frac{1}{10}
    \begin{pmatrix}
        -7 & 3\\
        -8 & 2
    \end{pmatrix}
    \qquad
    \kappa(A) = ||A||_1 \; ||A^{-1}||_1 =  10 \cdot \frac{15}{10} = 15
    \qquad
    \delta{A} = \frac{||\eps_A||_1}{||A||_1} = 0.002
\end{equation*}

Погрешность приближения
\begin{equation*}
    \delta{A^{-1}} = \frac{\kappa(A) \delta{A}}{1 - \kappa(A) \delta{A}}
    =
    \frac{15\cdot{0.002}}{1-15\cdot{0.002}} = \frac{0.03}{0.97} = 0.031
\end{equation*}