\documentclass{article}
\usepackage[margin=0.5in]{geometry}
\usepackage[utf8]{inputenc}   % поддержка UTF8
\usepackage[T2A]{fontenc}     % внутренняя T2A кодировка TeX
\usepackage[english, russian]{babel}
\usepackage{amsmath}
\usepackage{amssymb}


\begin{document}
\section*{Задание №6: Градиентный метод}

\begin{equation*}
    \vec{x}_{k+1} =
        \vec{x}_k
        - h_k \nabla f (\vec{x}_k).
\end{equation*}

Реализовать 3 способа выбора $h_k$:
\begin{enumerate}
    \item Находим $h_k$ из условия минимума (метод золотого сечения).
    \item Положить $h_k = \text{const}$.
    \item Использовать условие:
    \begin{equation*}
        \alpha \langle \nabla f(\vec{x}_k), \vec{x}_{k+1} - \vec{x}_k \rangle
        \leqslant f(\vec{x}_k) - f(\vec{x}_{k+1})
        \leqslant \beta \langle \nabla f(\vec{x}_k), \vec{x}_{k+1} - \vec{x}_k \rangle
    \end{equation*}
    c $0 < \alpha < \beta < 1$.
\end{enumerate}


\subsection*{Тестовые функции}

\begin{itemize}
    \item{}
        Выпуклая функция (квадратичная)
        \begin{equation*}
            f(x_1, x_2) = 100 \left(x_1 - x_2\right)^2
                          + 5 \sum_{j=2}^{n}\left(1 - x_j\right)^2.
        \end{equation*}
        Попробовать $n$ = 2, 3, 5, 10.
    \item{}
        Функция Розенброка
        \begin{equation*}
            f(x_1, x_2) = 100 \left(x_2 - x_1^2\right)^2
                          + 5\left(1 - x_1\right)^2.
        \end{equation*}
        \begin{equation*}
            \vec{x}^* =
            \begin{pmatrix}1\\1\end{pmatrix}, \qquad f^* = 0.
        \end{equation*}
        Принять $\vec{x}_0 = \vec{0}$.

\end{itemize}

Для $n = 2$ построить на плоскости $(x_1, x_2)$ траекторию
последовательных приближений $\{\vec{x}_k\}$ (соединить соседние
точки ломаной).

\end{document}