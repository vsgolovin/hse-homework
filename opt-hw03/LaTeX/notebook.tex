\documentclass[11pt]{article}

    \usepackage[utf8]{inputenc}
    \usepackage[english,russian]{babel}
    \usepackage[T2A]{fontenc}

    \usepackage[breakable]{tcolorbox}
    \usepackage{parskip} % Stop auto-indenting (to mimic markdown behaviour)
    
    \usepackage{iftex}
    \ifPDFTeX
    	\usepackage[T1]{fontenc}
    	\usepackage{mathpazo}
    \else
    	\usepackage{fontspec}
    \fi

    % Basic figure setup, for now with no caption control since it's done
    % automatically by Pandoc (which extracts ![](path) syntax from Markdown).
    \usepackage{graphicx}
    % Maintain compatibility with old templates. Remove in nbconvert 6.0
    \let\Oldincludegraphics\includegraphics
    % Ensure that by default, figures have no caption (until we provide a
    % proper Figure object with a Caption API and a way to capture that
    % in the conversion process - todo).
    \usepackage{caption}
    \DeclareCaptionFormat{nocaption}{}
    \captionsetup{format=nocaption,aboveskip=0pt,belowskip=0pt}

    \usepackage[Export]{adjustbox} % Used to constrain images to a maximum size
    \adjustboxset{max size={0.9\linewidth}{0.9\paperheight}}
    \usepackage{float}
    \floatplacement{figure}{H} % forces figures to be placed at the correct location
    \usepackage{xcolor} % Allow colors to be defined
    \usepackage{enumerate} % Needed for markdown enumerations to work
    \usepackage{geometry} % Used to adjust the document margins
    \usepackage{amsmath} % Equations
    \usepackage{amssymb} % Equations
    \usepackage{textcomp} % defines textquotesingle
    % Hack from http://tex.stackexchange.com/a/47451/13684:
    \AtBeginDocument{%
        \def\PYZsq{\textquotesingle}% Upright quotes in Pygmentized code
    }
    \usepackage{upquote} % Upright quotes for verbatim code
    \usepackage{eurosym} % defines \euro
    \usepackage[mathletters]{ucs} % Extended unicode (utf-8) support
    \usepackage{fancyvrb} % verbatim replacement that allows latex

    % The hyperref package gives us a pdf with properly built
    % internal navigation ('pdf bookmarks' for the table of contents,
    % internal cross-reference links, web links for URLs, etc.)
    \usepackage{hyperref}
    % The default LaTeX title has an obnoxious amount of whitespace. By default,
    % titling removes some of it. It also provides customization options.
    \usepackage{titling}
    \usepackage{longtable} % longtable support required by pandoc >1.10
    \usepackage{booktabs}  % table support for pandoc > 1.12.2
    \usepackage[inline]{enumitem} % IRkernel/repr support (it uses the enumerate* environment)
    \usepackage[normalem]{ulem} % ulem is needed to support strikethroughs (\sout)
                                % normalem makes italics be italics, not underlines
    \usepackage{mathrsfs}
    

    
    % Colors for the hyperref package
    \definecolor{urlcolor}{rgb}{0,.145,.698}
    \definecolor{linkcolor}{rgb}{.71,0.21,0.01}
    \definecolor{citecolor}{rgb}{.12,.54,.11}

    % ANSI colors
    \definecolor{ansi-black}{HTML}{3E424D}
    \definecolor{ansi-black-intense}{HTML}{282C36}
    \definecolor{ansi-red}{HTML}{E75C58}
    \definecolor{ansi-red-intense}{HTML}{B22B31}
    \definecolor{ansi-green}{HTML}{00A250}
    \definecolor{ansi-green-intense}{HTML}{007427}
    \definecolor{ansi-yellow}{HTML}{DDB62B}
    \definecolor{ansi-yellow-intense}{HTML}{B27D12}
    \definecolor{ansi-blue}{HTML}{208FFB}
    \definecolor{ansi-blue-intense}{HTML}{0065CA}
    \definecolor{ansi-magenta}{HTML}{D160C4}
    \definecolor{ansi-magenta-intense}{HTML}{A03196}
    \definecolor{ansi-cyan}{HTML}{60C6C8}
    \definecolor{ansi-cyan-intense}{HTML}{258F8F}
    \definecolor{ansi-white}{HTML}{C5C1B4}
    \definecolor{ansi-white-intense}{HTML}{A1A6B2}
    \definecolor{ansi-default-inverse-fg}{HTML}{FFFFFF}
    \definecolor{ansi-default-inverse-bg}{HTML}{000000}

    % commands and environments needed by pandoc snippets
    % extracted from the output of `pandoc -s`
    \providecommand{\tightlist}{%
      \setlength{\itemsep}{0pt}\setlength{\parskip}{0pt}}
    \DefineVerbatimEnvironment{Highlighting}{Verbatim}{commandchars=\\\{\}}
    % Add ',fontsize=\small' for more characters per line
    \newenvironment{Shaded}{}{}
    \newcommand{\KeywordTok}[1]{\textcolor[rgb]{0.00,0.44,0.13}{\textbf{{#1}}}}
    \newcommand{\DataTypeTok}[1]{\textcolor[rgb]{0.56,0.13,0.00}{{#1}}}
    \newcommand{\DecValTok}[1]{\textcolor[rgb]{0.25,0.63,0.44}{{#1}}}
    \newcommand{\BaseNTok}[1]{\textcolor[rgb]{0.25,0.63,0.44}{{#1}}}
    \newcommand{\FloatTok}[1]{\textcolor[rgb]{0.25,0.63,0.44}{{#1}}}
    \newcommand{\CharTok}[1]{\textcolor[rgb]{0.25,0.44,0.63}{{#1}}}
    \newcommand{\StringTok}[1]{\textcolor[rgb]{0.25,0.44,0.63}{{#1}}}
    \newcommand{\CommentTok}[1]{\textcolor[rgb]{0.38,0.63,0.69}{\textit{{#1}}}}
    \newcommand{\OtherTok}[1]{\textcolor[rgb]{0.00,0.44,0.13}{{#1}}}
    \newcommand{\AlertTok}[1]{\textcolor[rgb]{1.00,0.00,0.00}{\textbf{{#1}}}}
    \newcommand{\FunctionTok}[1]{\textcolor[rgb]{0.02,0.16,0.49}{{#1}}}
    \newcommand{\RegionMarkerTok}[1]{{#1}}
    \newcommand{\ErrorTok}[1]{\textcolor[rgb]{1.00,0.00,0.00}{\textbf{{#1}}}}
    \newcommand{\NormalTok}[1]{{#1}}
    
    % Additional commands for more recent versions of Pandoc
    \newcommand{\ConstantTok}[1]{\textcolor[rgb]{0.53,0.00,0.00}{{#1}}}
    \newcommand{\SpecialCharTok}[1]{\textcolor[rgb]{0.25,0.44,0.63}{{#1}}}
    \newcommand{\VerbatimStringTok}[1]{\textcolor[rgb]{0.25,0.44,0.63}{{#1}}}
    \newcommand{\SpecialStringTok}[1]{\textcolor[rgb]{0.73,0.40,0.53}{{#1}}}
    \newcommand{\ImportTok}[1]{{#1}}
    \newcommand{\DocumentationTok}[1]{\textcolor[rgb]{0.73,0.13,0.13}{\textit{{#1}}}}
    \newcommand{\AnnotationTok}[1]{\textcolor[rgb]{0.38,0.63,0.69}{\textbf{\textit{{#1}}}}}
    \newcommand{\CommentVarTok}[1]{\textcolor[rgb]{0.38,0.63,0.69}{\textbf{\textit{{#1}}}}}
    \newcommand{\VariableTok}[1]{\textcolor[rgb]{0.10,0.09,0.49}{{#1}}}
    \newcommand{\ControlFlowTok}[1]{\textcolor[rgb]{0.00,0.44,0.13}{\textbf{{#1}}}}
    \newcommand{\OperatorTok}[1]{\textcolor[rgb]{0.40,0.40,0.40}{{#1}}}
    \newcommand{\BuiltInTok}[1]{{#1}}
    \newcommand{\ExtensionTok}[1]{{#1}}
    \newcommand{\PreprocessorTok}[1]{\textcolor[rgb]{0.74,0.48,0.00}{{#1}}}
    \newcommand{\AttributeTok}[1]{\textcolor[rgb]{0.49,0.56,0.16}{{#1}}}
    \newcommand{\InformationTok}[1]{\textcolor[rgb]{0.38,0.63,0.69}{\textbf{\textit{{#1}}}}}
    \newcommand{\WarningTok}[1]{\textcolor[rgb]{0.38,0.63,0.69}{\textbf{\textit{{#1}}}}}
    
    
    % Define a nice break command that doesn't care if a line doesn't already
    % exist.
    \def\br{\hspace*{\fill} \\* }
    % Math Jax compatibility definitions
    \def\gt{>}
    \def\lt{<}
    \let\Oldtex\TeX
    \let\Oldlatex\LaTeX
    \renewcommand{\TeX}{\textrm{\Oldtex}}
    \renewcommand{\LaTeX}{\textrm{\Oldlatex}}
    % Document parameters
    % Document title
    \title{notebook}
    
    
    
    
    
% Pygments definitions
\makeatletter
\def\PY@reset{\let\PY@it=\relax \let\PY@bf=\relax%
    \let\PY@ul=\relax \let\PY@tc=\relax%
    \let\PY@bc=\relax \let\PY@ff=\relax}
\def\PY@tok#1{\csname PY@tok@#1\endcsname}
\def\PY@toks#1+{\ifx\relax#1\empty\else%
    \PY@tok{#1}\expandafter\PY@toks\fi}
\def\PY@do#1{\PY@bc{\PY@tc{\PY@ul{%
    \PY@it{\PY@bf{\PY@ff{#1}}}}}}}
\def\PY#1#2{\PY@reset\PY@toks#1+\relax+\PY@do{#2}}

\expandafter\def\csname PY@tok@w\endcsname{\def\PY@tc##1{\textcolor[rgb]{0.73,0.73,0.73}{##1}}}
\expandafter\def\csname PY@tok@c\endcsname{\let\PY@it=\textit\def\PY@tc##1{\textcolor[rgb]{0.25,0.50,0.50}{##1}}}
\expandafter\def\csname PY@tok@cp\endcsname{\def\PY@tc##1{\textcolor[rgb]{0.74,0.48,0.00}{##1}}}
\expandafter\def\csname PY@tok@k\endcsname{\let\PY@bf=\textbf\def\PY@tc##1{\textcolor[rgb]{0.00,0.50,0.00}{##1}}}
\expandafter\def\csname PY@tok@kp\endcsname{\def\PY@tc##1{\textcolor[rgb]{0.00,0.50,0.00}{##1}}}
\expandafter\def\csname PY@tok@kt\endcsname{\def\PY@tc##1{\textcolor[rgb]{0.69,0.00,0.25}{##1}}}
\expandafter\def\csname PY@tok@o\endcsname{\def\PY@tc##1{\textcolor[rgb]{0.40,0.40,0.40}{##1}}}
\expandafter\def\csname PY@tok@ow\endcsname{\let\PY@bf=\textbf\def\PY@tc##1{\textcolor[rgb]{0.67,0.13,1.00}{##1}}}
\expandafter\def\csname PY@tok@nb\endcsname{\def\PY@tc##1{\textcolor[rgb]{0.00,0.50,0.00}{##1}}}
\expandafter\def\csname PY@tok@nf\endcsname{\def\PY@tc##1{\textcolor[rgb]{0.00,0.00,1.00}{##1}}}
\expandafter\def\csname PY@tok@nc\endcsname{\let\PY@bf=\textbf\def\PY@tc##1{\textcolor[rgb]{0.00,0.00,1.00}{##1}}}
\expandafter\def\csname PY@tok@nn\endcsname{\let\PY@bf=\textbf\def\PY@tc##1{\textcolor[rgb]{0.00,0.00,1.00}{##1}}}
\expandafter\def\csname PY@tok@ne\endcsname{\let\PY@bf=\textbf\def\PY@tc##1{\textcolor[rgb]{0.82,0.25,0.23}{##1}}}
\expandafter\def\csname PY@tok@nv\endcsname{\def\PY@tc##1{\textcolor[rgb]{0.10,0.09,0.49}{##1}}}
\expandafter\def\csname PY@tok@no\endcsname{\def\PY@tc##1{\textcolor[rgb]{0.53,0.00,0.00}{##1}}}
\expandafter\def\csname PY@tok@nl\endcsname{\def\PY@tc##1{\textcolor[rgb]{0.63,0.63,0.00}{##1}}}
\expandafter\def\csname PY@tok@ni\endcsname{\let\PY@bf=\textbf\def\PY@tc##1{\textcolor[rgb]{0.60,0.60,0.60}{##1}}}
\expandafter\def\csname PY@tok@na\endcsname{\def\PY@tc##1{\textcolor[rgb]{0.49,0.56,0.16}{##1}}}
\expandafter\def\csname PY@tok@nt\endcsname{\let\PY@bf=\textbf\def\PY@tc##1{\textcolor[rgb]{0.00,0.50,0.00}{##1}}}
\expandafter\def\csname PY@tok@nd\endcsname{\def\PY@tc##1{\textcolor[rgb]{0.67,0.13,1.00}{##1}}}
\expandafter\def\csname PY@tok@s\endcsname{\def\PY@tc##1{\textcolor[rgb]{0.73,0.13,0.13}{##1}}}
\expandafter\def\csname PY@tok@sd\endcsname{\let\PY@it=\textit\def\PY@tc##1{\textcolor[rgb]{0.73,0.13,0.13}{##1}}}
\expandafter\def\csname PY@tok@si\endcsname{\let\PY@bf=\textbf\def\PY@tc##1{\textcolor[rgb]{0.73,0.40,0.53}{##1}}}
\expandafter\def\csname PY@tok@se\endcsname{\let\PY@bf=\textbf\def\PY@tc##1{\textcolor[rgb]{0.73,0.40,0.13}{##1}}}
\expandafter\def\csname PY@tok@sr\endcsname{\def\PY@tc##1{\textcolor[rgb]{0.73,0.40,0.53}{##1}}}
\expandafter\def\csname PY@tok@ss\endcsname{\def\PY@tc##1{\textcolor[rgb]{0.10,0.09,0.49}{##1}}}
\expandafter\def\csname PY@tok@sx\endcsname{\def\PY@tc##1{\textcolor[rgb]{0.00,0.50,0.00}{##1}}}
\expandafter\def\csname PY@tok@m\endcsname{\def\PY@tc##1{\textcolor[rgb]{0.40,0.40,0.40}{##1}}}
\expandafter\def\csname PY@tok@gh\endcsname{\let\PY@bf=\textbf\def\PY@tc##1{\textcolor[rgb]{0.00,0.00,0.50}{##1}}}
\expandafter\def\csname PY@tok@gu\endcsname{\let\PY@bf=\textbf\def\PY@tc##1{\textcolor[rgb]{0.50,0.00,0.50}{##1}}}
\expandafter\def\csname PY@tok@gd\endcsname{\def\PY@tc##1{\textcolor[rgb]{0.63,0.00,0.00}{##1}}}
\expandafter\def\csname PY@tok@gi\endcsname{\def\PY@tc##1{\textcolor[rgb]{0.00,0.63,0.00}{##1}}}
\expandafter\def\csname PY@tok@gr\endcsname{\def\PY@tc##1{\textcolor[rgb]{1.00,0.00,0.00}{##1}}}
\expandafter\def\csname PY@tok@ge\endcsname{\let\PY@it=\textit}
\expandafter\def\csname PY@tok@gs\endcsname{\let\PY@bf=\textbf}
\expandafter\def\csname PY@tok@gp\endcsname{\let\PY@bf=\textbf\def\PY@tc##1{\textcolor[rgb]{0.00,0.00,0.50}{##1}}}
\expandafter\def\csname PY@tok@go\endcsname{\def\PY@tc##1{\textcolor[rgb]{0.53,0.53,0.53}{##1}}}
\expandafter\def\csname PY@tok@gt\endcsname{\def\PY@tc##1{\textcolor[rgb]{0.00,0.27,0.87}{##1}}}
\expandafter\def\csname PY@tok@err\endcsname{\def\PY@bc##1{\setlength{\fboxsep}{0pt}\fcolorbox[rgb]{1.00,0.00,0.00}{1,1,1}{\strut ##1}}}
\expandafter\def\csname PY@tok@kc\endcsname{\let\PY@bf=\textbf\def\PY@tc##1{\textcolor[rgb]{0.00,0.50,0.00}{##1}}}
\expandafter\def\csname PY@tok@kd\endcsname{\let\PY@bf=\textbf\def\PY@tc##1{\textcolor[rgb]{0.00,0.50,0.00}{##1}}}
\expandafter\def\csname PY@tok@kn\endcsname{\let\PY@bf=\textbf\def\PY@tc##1{\textcolor[rgb]{0.00,0.50,0.00}{##1}}}
\expandafter\def\csname PY@tok@kr\endcsname{\let\PY@bf=\textbf\def\PY@tc##1{\textcolor[rgb]{0.00,0.50,0.00}{##1}}}
\expandafter\def\csname PY@tok@bp\endcsname{\def\PY@tc##1{\textcolor[rgb]{0.00,0.50,0.00}{##1}}}
\expandafter\def\csname PY@tok@fm\endcsname{\def\PY@tc##1{\textcolor[rgb]{0.00,0.00,1.00}{##1}}}
\expandafter\def\csname PY@tok@vc\endcsname{\def\PY@tc##1{\textcolor[rgb]{0.10,0.09,0.49}{##1}}}
\expandafter\def\csname PY@tok@vg\endcsname{\def\PY@tc##1{\textcolor[rgb]{0.10,0.09,0.49}{##1}}}
\expandafter\def\csname PY@tok@vi\endcsname{\def\PY@tc##1{\textcolor[rgb]{0.10,0.09,0.49}{##1}}}
\expandafter\def\csname PY@tok@vm\endcsname{\def\PY@tc##1{\textcolor[rgb]{0.10,0.09,0.49}{##1}}}
\expandafter\def\csname PY@tok@sa\endcsname{\def\PY@tc##1{\textcolor[rgb]{0.73,0.13,0.13}{##1}}}
\expandafter\def\csname PY@tok@sb\endcsname{\def\PY@tc##1{\textcolor[rgb]{0.73,0.13,0.13}{##1}}}
\expandafter\def\csname PY@tok@sc\endcsname{\def\PY@tc##1{\textcolor[rgb]{0.73,0.13,0.13}{##1}}}
\expandafter\def\csname PY@tok@dl\endcsname{\def\PY@tc##1{\textcolor[rgb]{0.73,0.13,0.13}{##1}}}
\expandafter\def\csname PY@tok@s2\endcsname{\def\PY@tc##1{\textcolor[rgb]{0.73,0.13,0.13}{##1}}}
\expandafter\def\csname PY@tok@sh\endcsname{\def\PY@tc##1{\textcolor[rgb]{0.73,0.13,0.13}{##1}}}
\expandafter\def\csname PY@tok@s1\endcsname{\def\PY@tc##1{\textcolor[rgb]{0.73,0.13,0.13}{##1}}}
\expandafter\def\csname PY@tok@mb\endcsname{\def\PY@tc##1{\textcolor[rgb]{0.40,0.40,0.40}{##1}}}
\expandafter\def\csname PY@tok@mf\endcsname{\def\PY@tc##1{\textcolor[rgb]{0.40,0.40,0.40}{##1}}}
\expandafter\def\csname PY@tok@mh\endcsname{\def\PY@tc##1{\textcolor[rgb]{0.40,0.40,0.40}{##1}}}
\expandafter\def\csname PY@tok@mi\endcsname{\def\PY@tc##1{\textcolor[rgb]{0.40,0.40,0.40}{##1}}}
\expandafter\def\csname PY@tok@il\endcsname{\def\PY@tc##1{\textcolor[rgb]{0.40,0.40,0.40}{##1}}}
\expandafter\def\csname PY@tok@mo\endcsname{\def\PY@tc##1{\textcolor[rgb]{0.40,0.40,0.40}{##1}}}
\expandafter\def\csname PY@tok@ch\endcsname{\let\PY@it=\textit\def\PY@tc##1{\textcolor[rgb]{0.25,0.50,0.50}{##1}}}
\expandafter\def\csname PY@tok@cm\endcsname{\let\PY@it=\textit\def\PY@tc##1{\textcolor[rgb]{0.25,0.50,0.50}{##1}}}
\expandafter\def\csname PY@tok@cpf\endcsname{\let\PY@it=\textit\def\PY@tc##1{\textcolor[rgb]{0.25,0.50,0.50}{##1}}}
\expandafter\def\csname PY@tok@c1\endcsname{\let\PY@it=\textit\def\PY@tc##1{\textcolor[rgb]{0.25,0.50,0.50}{##1}}}
\expandafter\def\csname PY@tok@cs\endcsname{\let\PY@it=\textit\def\PY@tc##1{\textcolor[rgb]{0.25,0.50,0.50}{##1}}}

\def\PYZbs{\char`\\}
\def\PYZus{\char`\_}
\def\PYZob{\char`\{}
\def\PYZcb{\char`\}}
\def\PYZca{\char`\^}
\def\PYZam{\char`\&}
\def\PYZlt{\char`\<}
\def\PYZgt{\char`\>}
\def\PYZsh{\char`\#}
\def\PYZpc{\char`\%}
\def\PYZdl{\char`\$}
\def\PYZhy{\char`\-}
\def\PYZsq{\char`\'}
\def\PYZdq{\char`\"}
\def\PYZti{\char`\~}
% for compatibility with earlier versions
\def\PYZat{@}
\def\PYZlb{[}
\def\PYZrb{]}
\makeatother


    % For linebreaks inside Verbatim environment from package fancyvrb. 
    \makeatletter
        \newbox\Wrappedcontinuationbox 
        \newbox\Wrappedvisiblespacebox 
        \newcommand*\Wrappedvisiblespace {\textcolor{red}{\textvisiblespace}} 
        \newcommand*\Wrappedcontinuationsymbol {\textcolor{red}{\llap{\tiny$\m@th\hookrightarrow$}}} 
        \newcommand*\Wrappedcontinuationindent {3ex } 
        \newcommand*\Wrappedafterbreak {\kern\Wrappedcontinuationindent\copy\Wrappedcontinuationbox} 
        % Take advantage of the already applied Pygments mark-up to insert 
        % potential linebreaks for TeX processing. 
        %        {, <, #, %, $, ' and ": go to next line. 
        %        _, }, ^, &, >, - and ~: stay at end of broken line. 
        % Use of \textquotesingle for straight quote. 
        \newcommand*\Wrappedbreaksatspecials {% 
            \def\PYGZus{\discretionary{\char`\_}{\Wrappedafterbreak}{\char`\_}}% 
            \def\PYGZob{\discretionary{}{\Wrappedafterbreak\char`\{}{\char`\{}}% 
            \def\PYGZcb{\discretionary{\char`\}}{\Wrappedafterbreak}{\char`\}}}% 
            \def\PYGZca{\discretionary{\char`\^}{\Wrappedafterbreak}{\char`\^}}% 
            \def\PYGZam{\discretionary{\char`\&}{\Wrappedafterbreak}{\char`\&}}% 
            \def\PYGZlt{\discretionary{}{\Wrappedafterbreak\char`\<}{\char`\<}}% 
            \def\PYGZgt{\discretionary{\char`\>}{\Wrappedafterbreak}{\char`\>}}% 
            \def\PYGZsh{\discretionary{}{\Wrappedafterbreak\char`\#}{\char`\#}}% 
            \def\PYGZpc{\discretionary{}{\Wrappedafterbreak\char`\%}{\char`\%}}% 
            \def\PYGZdl{\discretionary{}{\Wrappedafterbreak\char`\$}{\char`\$}}% 
            \def\PYGZhy{\discretionary{\char`\-}{\Wrappedafterbreak}{\char`\-}}% 
            \def\PYGZsq{\discretionary{}{\Wrappedafterbreak\textquotesingle}{\textquotesingle}}% 
            \def\PYGZdq{\discretionary{}{\Wrappedafterbreak\char`\"}{\char`\"}}% 
            \def\PYGZti{\discretionary{\char`\~}{\Wrappedafterbreak}{\char`\~}}% 
        } 
        % Some characters . , ; ? ! / are not pygmentized. 
        % This macro makes them "active" and they will insert potential linebreaks 
        \newcommand*\Wrappedbreaksatpunct {% 
            \lccode`\~`\.\lowercase{\def~}{\discretionary{\hbox{\char`\.}}{\Wrappedafterbreak}{\hbox{\char`\.}}}% 
            \lccode`\~`\,\lowercase{\def~}{\discretionary{\hbox{\char`\,}}{\Wrappedafterbreak}{\hbox{\char`\,}}}% 
            \lccode`\~`\;\lowercase{\def~}{\discretionary{\hbox{\char`\;}}{\Wrappedafterbreak}{\hbox{\char`\;}}}% 
            \lccode`\~`\:\lowercase{\def~}{\discretionary{\hbox{\char`\:}}{\Wrappedafterbreak}{\hbox{\char`\:}}}% 
            \lccode`\~`\?\lowercase{\def~}{\discretionary{\hbox{\char`\?}}{\Wrappedafterbreak}{\hbox{\char`\?}}}% 
            \lccode`\~`\!\lowercase{\def~}{\discretionary{\hbox{\char`\!}}{\Wrappedafterbreak}{\hbox{\char`\!}}}% 
            \lccode`\~`\/\lowercase{\def~}{\discretionary{\hbox{\char`\/}}{\Wrappedafterbreak}{\hbox{\char`\/}}}% 
            \catcode`\.\active
            \catcode`\,\active 
            \catcode`\;\active
            \catcode`\:\active
            \catcode`\?\active
            \catcode`\!\active
            \catcode`\/\active 
            \lccode`\~`\~ 	
        }
    \makeatother

    \let\OriginalVerbatim=\Verbatim
    \makeatletter
    \renewcommand{\Verbatim}[1][1]{%
        %\parskip\z@skip
        \sbox\Wrappedcontinuationbox {\Wrappedcontinuationsymbol}%
        \sbox\Wrappedvisiblespacebox {\FV@SetupFont\Wrappedvisiblespace}%
        \def\FancyVerbFormatLine ##1{\hsize\linewidth
            \vtop{\raggedright\hyphenpenalty\z@\exhyphenpenalty\z@
                \doublehyphendemerits\z@\finalhyphendemerits\z@
                \strut ##1\strut}%
        }%
        % If the linebreak is at a space, the latter will be displayed as visible
        % space at end of first line, and a continuation symbol starts next line.
        % Stretch/shrink are however usually zero for typewriter font.
        \def\FV@Space {%
            \nobreak\hskip\z@ plus\fontdimen3\font minus\fontdimen4\font
            \discretionary{\copy\Wrappedvisiblespacebox}{\Wrappedafterbreak}
            {\kern\fontdimen2\font}%
        }%
        
        % Allow breaks at special characters using \PYG... macros.
        \Wrappedbreaksatspecials
        % Breaks at punctuation characters . , ; ? ! and / need catcode=\active 	
        \OriginalVerbatim[#1,codes*=\Wrappedbreaksatpunct]%
    }
    \makeatother

    % Exact colors from NB
    \definecolor{incolor}{HTML}{303F9F}
    \definecolor{outcolor}{HTML}{D84315}
    \definecolor{cellborder}{HTML}{CFCFCF}
    \definecolor{cellbackground}{HTML}{F7F7F7}
    
    % prompt
    \makeatletter
    \newcommand{\boxspacing}{\kern\kvtcb@left@rule\kern\kvtcb@boxsep}
    \makeatother
    \newcommand{\prompt}[4]{
        \ttfamily\llap{{\color{#2}[#3]:\hspace{3pt}#4}}\vspace{-\baselineskip}
    }
    

    
    % Prevent overflowing lines due to hard-to-break entities
    \sloppy 
    % Setup hyperref package
    \hypersetup{
      breaklinks=true,  % so long urls are correctly broken across lines
      colorlinks=true,
      urlcolor=urlcolor,
      linkcolor=linkcolor,
      citecolor=citecolor,
      }
    % Slightly bigger margins than the latex defaults
    
    \geometry{verbose,tmargin=1in,bmargin=1in,lmargin=1in,rmargin=1in}



\begin{document}

\title{Домашняя работа №6}
\author{Головин Вячеслав Сергеевич (ВШЭ, МОАД, 5 курс)}
\maketitle

    
    \begin{tcolorbox}[breakable, size=fbox, boxrule=1pt, pad at break*=1mm,colback=cellbackground, colframe=cellborder]
\prompt{In}{incolor}{1}{\boxspacing}
\begin{Verbatim}[commandchars=\\\{\}]
\PY{o}{\PYZpc{}}\PY{k}{matplotlib} inline

\PY{k+kn}{import} \PY{n+nn}{numpy} \PY{k}{as} \PY{n+nn}{np}
\PY{k+kn}{import} \PY{n+nn}{matplotlib}\PY{n+nn}{.}\PY{n+nn}{pyplot} \PY{k}{as} \PY{n+nn}{plt}

\PY{n}{plt}\PY{o}{.}\PY{n}{rc}\PY{p}{(}\PY{l+s+s1}{\PYZsq{}}\PY{l+s+s1}{figure}\PY{l+s+s1}{\PYZsq{}}\PY{p}{,} \PY{n}{dpi}\PY{o}{=}\PY{l+m+mi}{300}\PY{p}{)}
\end{Verbatim}
\end{tcolorbox}

    Будем рассматривать две функции:

\begin{enumerate}
\def\labelenumi{\arabic{enumi}.}
\tightlist
\item
  квадратичная функция \begin{equation*}
   f(x_1, x_2) = 100 \left(x_1 - x_2\right)^2
                   + 5 \sum_{j=2}^{n}\left(1 - x_j\right)^2;
  \end{equation*}
\item
  функция Розенброка \begin{equation*}
   f(x_1, x_2) = 100 \left(x_2 - x_1^2\right)^2
                   + 5\left(1 - x_1\right)^2.
  \end{equation*}
\end{enumerate}

    \begin{tcolorbox}[breakable, size=fbox, boxrule=1pt, pad at break*=1mm,colback=cellbackground, colframe=cellborder]
\prompt{In}{incolor}{2}{\boxspacing}
\begin{Verbatim}[commandchars=\\\{\}]
\PY{k}{def} \PY{n+nf}{quadratic}\PY{p}{(}\PY{o}{*}\PY{n}{args}\PY{p}{)}\PY{p}{:}
    \PY{k}{return} \PY{p}{(}\PY{l+m+mi}{100} \PY{o}{*} \PY{p}{(}\PY{n}{args}\PY{p}{[}\PY{l+m+mi}{0}\PY{p}{]} \PY{o}{\PYZhy{}} \PY{n}{args}\PY{p}{[}\PY{l+m+mi}{1}\PY{p}{]}\PY{p}{)}\PY{o}{*}\PY{o}{*}\PY{l+m+mi}{2}
            \PY{o}{+} \PY{l+m+mi}{5} \PY{o}{*} \PY{n+nb}{sum}\PY{p}{(}\PY{p}{(}\PY{l+m+mi}{1} \PY{o}{\PYZhy{}} \PY{n}{xi}\PY{p}{)}\PY{o}{*}\PY{o}{*}\PY{l+m+mi}{2} \PY{k}{for} \PY{n}{xi} \PY{o+ow}{in} \PY{n}{args}\PY{p}{[}\PY{l+m+mi}{1}\PY{p}{:}\PY{p}{]}\PY{p}{)}\PY{p}{)}

\PY{k}{def} \PY{n+nf}{quadratic\PYZus{}derivatives}\PY{p}{(}\PY{o}{*}\PY{n}{args}\PY{p}{)}\PY{p}{:}
    \PY{n}{derivatives} \PY{o}{=} \PY{p}{[}
        \PY{l+m+mi}{200} \PY{o}{*} \PY{p}{(}\PY{n}{args}\PY{p}{[}\PY{l+m+mi}{0}\PY{p}{]} \PY{o}{\PYZhy{}} \PY{n}{args}\PY{p}{[}\PY{l+m+mi}{1}\PY{p}{]}\PY{p}{)}\PY{p}{,}
        \PY{o}{\PYZhy{}}\PY{l+m+mi}{200} \PY{o}{*} \PY{p}{(}\PY{n}{args}\PY{p}{[}\PY{l+m+mi}{0}\PY{p}{]} \PY{o}{\PYZhy{}} \PY{n}{args}\PY{p}{[}\PY{l+m+mi}{1}\PY{p}{]}\PY{p}{)} \PY{o}{\PYZhy{}} \PY{l+m+mi}{10} \PY{o}{*} \PY{p}{(}\PY{l+m+mi}{1} \PY{o}{\PYZhy{}} \PY{n}{args}\PY{p}{[}\PY{l+m+mi}{1}\PY{p}{]}\PY{p}{)}
    \PY{p}{]}
    \PY{k}{return} \PY{n}{derivatives} \PY{o}{+} \PY{p}{[}\PY{o}{\PYZhy{}}\PY{l+m+mi}{10} \PY{o}{*} \PY{p}{(}\PY{l+m+mi}{1} \PY{o}{\PYZhy{}} \PY{n}{xi}\PY{p}{)} \PY{k}{for} \PY{n}{xi} \PY{o+ow}{in} \PY{n}{args}\PY{p}{[}\PY{l+m+mi}{2}\PY{p}{:}\PY{p}{]}\PY{p}{]}

\PY{k}{def} \PY{n+nf}{rosenbrock}\PY{p}{(}\PY{n}{x1}\PY{p}{,} \PY{n}{x2}\PY{p}{)}\PY{p}{:}
    \PY{k}{return} \PY{l+m+mi}{100} \PY{o}{*} \PY{p}{(}\PY{n}{x2} \PY{o}{\PYZhy{}} \PY{n}{x1}\PY{o}{*}\PY{o}{*}\PY{l+m+mi}{2}\PY{p}{)}\PY{o}{*}\PY{o}{*}\PY{l+m+mi}{2} \PY{o}{+} \PY{l+m+mi}{5} \PY{o}{*} \PY{p}{(}\PY{l+m+mi}{1} \PY{o}{\PYZhy{}} \PY{n}{x1}\PY{p}{)}\PY{o}{*}\PY{o}{*}\PY{l+m+mi}{2}

\PY{k}{def} \PY{n+nf}{rosenbrock\PYZus{}derivatives}\PY{p}{(}\PY{n}{x1}\PY{p}{,} \PY{n}{x2}\PY{p}{)}\PY{p}{:}
    \PY{n}{t} \PY{o}{=} \PY{n}{x2} \PY{o}{\PYZhy{}} \PY{n}{x1}\PY{o}{*}\PY{o}{*}\PY{l+m+mi}{2}
    \PY{n}{df\PYZus{}dx1} \PY{o}{=} \PY{o}{\PYZhy{}}\PY{l+m+mi}{400} \PY{o}{*} \PY{n}{x1} \PY{o}{*} \PY{n}{t} \PY{o}{+} \PY{l+m+mi}{5} \PY{o}{*} \PY{p}{(}\PY{n}{x1} \PY{o}{\PYZhy{}} \PY{l+m+mi}{1}\PY{p}{)}
    \PY{n}{df\PYZus{}dx2} \PY{o}{=} \PY{l+m+mi}{200} \PY{o}{*} \PY{n}{t}
    \PY{k}{return} \PY{p}{[}\PY{n}{df\PYZus{}dx1}\PY{p}{,} \PY{n}{df\PYZus{}dx2}\PY{p}{]}
\end{Verbatim}
\end{tcolorbox}

\newpage
    Построим их графики:

    \begin{tcolorbox}[breakable, size=fbox, boxrule=1pt, pad at break*=1mm,colback=cellbackground, colframe=cellborder]
\prompt{In}{incolor}{3}{\boxspacing}
\begin{Verbatim}[commandchars=\\\{\}]
\PY{n}{mesh} \PY{o}{=} \PY{n}{np}\PY{o}{.}\PY{n}{meshgrid}\PY{p}{(}\PY{n}{np}\PY{o}{.}\PY{n}{linspace}\PY{p}{(}\PY{o}{\PYZhy{}}\PY{l+m+mi}{2}\PY{p}{,} \PY{l+m+mi}{3}\PY{p}{)}\PY{p}{,} \PY{n}{np}\PY{o}{.}\PY{n}{linspace}\PY{p}{(}\PY{o}{\PYZhy{}}\PY{l+m+mi}{2}\PY{p}{,} \PY{l+m+mi}{3}\PY{p}{)}\PY{p}{)}
\PY{n}{fig} \PY{o}{=} \PY{n}{plt}\PY{o}{.}\PY{n}{figure}\PY{p}{(}\PY{n}{figsize}\PY{o}{=}\PY{p}{(}\PY{l+m+mi}{9}\PY{p}{,} \PY{l+m+mi}{4}\PY{p}{)}\PY{p}{)}
\PY{n}{ax} \PY{o}{=} \PY{n}{fig}\PY{o}{.}\PY{n}{add\PYZus{}subplot}\PY{p}{(}\PY{l+m+mi}{121}\PY{p}{,} \PY{n}{projection}\PY{o}{=}\PY{l+s+s1}{\PYZsq{}}\PY{l+s+s1}{3d}\PY{l+s+s1}{\PYZsq{}}\PY{p}{,} \PY{n}{azim}\PY{o}{=}\PY{o}{\PYZhy{}}\PY{l+m+mi}{120}\PY{p}{)}
\PY{n}{ax}\PY{o}{.}\PY{n}{plot\PYZus{}surface}\PY{p}{(}\PY{o}{*}\PY{n}{mesh}\PY{p}{,} \PY{n}{quadratic}\PY{p}{(}\PY{o}{*}\PY{n}{mesh}\PY{p}{)}\PY{p}{)}
\PY{n}{ax2} \PY{o}{=} \PY{n}{fig}\PY{o}{.}\PY{n}{add\PYZus{}subplot}\PY{p}{(}\PY{l+m+mi}{122}\PY{p}{,} \PY{n}{projection}\PY{o}{=}\PY{l+s+s1}{\PYZsq{}}\PY{l+s+s1}{3d}\PY{l+s+s1}{\PYZsq{}}\PY{p}{,} \PY{n}{azim}\PY{o}{=}\PY{o}{\PYZhy{}}\PY{l+m+mi}{120}\PY{p}{)}
\PY{n}{ax2}\PY{o}{.}\PY{n}{plot\PYZus{}surface}\PY{p}{(}\PY{o}{*}\PY{n}{mesh}\PY{p}{,} \PY{n}{rosenbrock}\PY{p}{(}\PY{o}{*}\PY{n}{mesh}\PY{p}{)}\PY{p}{)}
\end{Verbatim}
\end{tcolorbox}

            \begin{tcolorbox}[breakable, size=fbox, boxrule=.5pt, pad at break*=1mm, opacityfill=0]
\prompt{Out}{outcolor}{3}{\boxspacing}
\begin{Verbatim}[commandchars=\\\{\}]
<mpl\_toolkits.mplot3d.art3d.Poly3DCollection at 0x7fa4039b4610>
\end{Verbatim}
\end{tcolorbox}
        
    \begin{center}
    \adjustimage{max size={0.9\linewidth}{0.9\paperheight}}{output_4_1.png}
    \end{center}
    { \hspace*{\fill} \\}
    
    Далее реализуем метод градиентного спуска для поиска минимума. На каждой
итерации этого метода текущее решение \(\vec{x}\) обновляется по формуле
\begin{equation*}
    \vec{x} \leftarrow \vec{x} - h \nabla f(\vec{x}).
\end{equation*}

Значение параметра \(h\) может выбираться тремя способами:
\begin{enumerate}
\item \(h = \text{argmin}{f(\vec{x} - h \nabla \vec{x})}\),
\item \(h = \text{const}\),
\item \(\alpha \langle \nabla f(\vec{x}_k), \vec{x}_{k+1} - \vec{x}_k \rangle \leqslant f(\vec{x}_k) - f(\vec{x}_{k+1}) \leqslant \beta \langle \nabla f(\vec{x}_k), \vec{x}_{k+1} - \vec{x}_k \rangle\),
где \(\vec{x}_{k+1} = \vec{x}_k - h \nabla f(\vec{x}_k)\).
\end{enumerate}

В первом случае для минимизации используется метод золотого сечения.
Из-за этого может возникать следующая проблема: минимизируемая функция
от \(h\) может иметь несколько локальных минимумов, а метод золотого
сечения найдёт один из них, который может не являться глобальным. Более
того, найденное значение \(h\) может привести к увеличению значения в
новой точке, т.е. \(f(\vec{x} + h \nabla f(\vec{x})) > f(\vec{x})\).
Поэтому после поиска \(h\) мы дополнительно проверяем это условие. Если
оно выполняется, то мы ищем новое (меньшее) значение \(h\) на интервале
\([0, h]\) (вместо изначального \([0, 1]\)).

    \begin{tcolorbox}[breakable, size=fbox, boxrule=1pt, pad at break*=1mm,colback=cellbackground, colframe=cellborder]
\prompt{In}{incolor}{4}{\boxspacing}
\begin{Verbatim}[commandchars=\\\{\}]
\PY{k}{def} \PY{n+nf}{golden\PYZus{}section\PYZus{}search}\PY{p}{(}\PY{n}{f}\PY{p}{,} \PY{n}{a}\PY{p}{,} \PY{n}{b}\PY{p}{,} \PY{n}{atol}\PY{o}{=}\PY{k+kc}{None}\PY{p}{)}\PY{p}{:}
    \PY{l+s+sd}{\PYZdq{}\PYZdq{}\PYZdq{}}
\PY{l+s+sd}{    Найти минимум функции `f` на интервале [`a`, `b`] с точностью `atol`}
\PY{l+s+sd}{    методом золотого сечения. Точность по умолчанию `1e\PYZhy{}4 * (b \PYZhy{} a)`.}
\PY{l+s+sd}{    \PYZdq{}\PYZdq{}\PYZdq{}}
    \PY{c+c1}{\PYZsh{} точность по умолчанию}
    \PY{k}{if} \PY{n}{atol} \PY{o+ow}{is} \PY{k+kc}{None}\PY{p}{:}
        \PY{n}{atol} \PY{o}{=} \PY{p}{(}\PY{n}{b} \PY{o}{\PYZhy{}} \PY{n}{a}\PY{p}{)} \PY{o}{*} \PY{l+m+mf}{1e\PYZhy{}4}
    \PY{c+c1}{\PYZsh{} коэффициент для разбиения отрезка}
    \PY{n}{phi} \PY{o}{=} \PY{p}{(}\PY{n}{np}\PY{o}{.}\PY{n}{sqrt}\PY{p}{(}\PY{l+m+mi}{5}\PY{p}{)} \PY{o}{\PYZhy{}} \PY{l+m+mi}{1}\PY{p}{)} \PY{o}{/} \PY{l+m+mi}{2}

    \PY{c+c1}{\PYZsh{} начальное разбиение}
    \PY{n}{c} \PY{o}{=} \PY{n}{b} \PY{o}{\PYZhy{}} \PY{p}{(}\PY{n}{b} \PY{o}{\PYZhy{}} \PY{n}{a}\PY{p}{)} \PY{o}{*} \PY{n}{phi}
    \PY{n}{f\PYZus{}c} \PY{o}{=} \PY{n}{f}\PY{p}{(}\PY{n}{c}\PY{p}{)}
    \PY{n}{d}\PY{p}{,} \PY{n}{f\PYZus{}d} \PY{o}{=} \PY{k+kc}{None}\PY{p}{,} \PY{k+kc}{None}

    \PY{k}{while} \PY{p}{(}\PY{n}{b} \PY{o}{\PYZhy{}} \PY{n}{a}\PY{p}{)} \PY{o}{\PYZgt{}} \PY{n}{atol}\PY{p}{:}
        \PY{c+c1}{\PYZsh{} новая точка}
        \PY{k}{if} \PY{n}{c} \PY{o+ow}{is} \PY{k+kc}{None}\PY{p}{:}
            \PY{n}{c} \PY{o}{=} \PY{n}{b} \PY{o}{\PYZhy{}} \PY{p}{(}\PY{n}{b} \PY{o}{\PYZhy{}} \PY{n}{a}\PY{p}{)} \PY{o}{*} \PY{n}{phi}
            \PY{n}{f\PYZus{}c} \PY{o}{=} \PY{n}{f}\PY{p}{(}\PY{n}{c}\PY{p}{)}
        \PY{k}{else}\PY{p}{:}
            \PY{k}{assert} \PY{n}{d} \PY{o+ow}{is} \PY{k+kc}{None}
            \PY{n}{d} \PY{o}{=} \PY{n}{a} \PY{o}{+} \PY{p}{(}\PY{n}{b} \PY{o}{\PYZhy{}} \PY{n}{a}\PY{p}{)} \PY{o}{*} \PY{n}{phi}
            \PY{n}{f\PYZus{}d} \PY{o}{=} \PY{n}{f}\PY{p}{(}\PY{n}{d}\PY{p}{)}

        \PY{c+c1}{\PYZsh{} выбор нового интервала}
        \PY{k}{if} \PY{n}{f\PYZus{}c} \PY{o}{\PYZlt{}} \PY{n}{f\PYZus{}d}\PY{p}{:}
            \PY{n}{b}\PY{p}{,} \PY{n}{d} \PY{o}{=} \PY{n}{d}\PY{p}{,} \PY{n}{c}
            \PY{n}{f\PYZus{}d} \PY{o}{=} \PY{n}{f\PYZus{}c}
            \PY{n}{c} \PY{o}{=} \PY{k+kc}{None}
        \PY{k}{else}\PY{p}{:}
            \PY{n}{a}\PY{p}{,} \PY{n}{c} \PY{o}{=} \PY{n}{c}\PY{p}{,} \PY{n}{d}
            \PY{n}{f\PYZus{}c} \PY{o}{=} \PY{n}{f\PYZus{}d}
            \PY{n}{d} \PY{o}{=} \PY{k+kc}{None}

    \PY{k}{return} \PY{n}{a} \PY{o}{+} \PY{p}{(}\PY{n}{b} \PY{o}{\PYZhy{}} \PY{n}{a}\PY{p}{)} \PY{o}{/} \PY{l+m+mi}{2}
\end{Verbatim}
\end{tcolorbox}

    \begin{tcolorbox}[breakable, size=fbox, boxrule=1pt, pad at break*=1mm,colback=cellbackground, colframe=cellborder]
\prompt{In}{incolor}{5}{\boxspacing}
\begin{Verbatim}[commandchars=\\\{\}]
\PY{k}{def} \PY{n+nf}{gradient\PYZus{}descent}\PY{p}{(}\PY{n}{f}\PY{p}{,} \PY{n}{grad\PYZus{}f}\PY{p}{,} \PY{n}{x0}\PY{p}{,} \PY{n}{h\PYZus{}method}\PY{o}{=}\PY{l+s+s1}{\PYZsq{}}\PY{l+s+s1}{min}\PY{l+s+s1}{\PYZsq{}}\PY{p}{,} \PY{n}{grad\PYZus{}min}\PY{o}{=}\PY{l+m+mf}{1e\PYZhy{}4}\PY{p}{,}
                     \PY{n}{return\PYZus{}solutions}\PY{o}{=}\PY{k+kc}{False}\PY{p}{)}\PY{p}{:}
    \PY{l+s+sd}{\PYZdq{}\PYZdq{}\PYZdq{}}
\PY{l+s+sd}{    Найти минимум функции `f` с градентом `grad\PYZus{}f` с помощью метода}
\PY{l+s+sd}{    градиентного спуска. Параметр `h\PYZus{}method` определяет способ выбора}
\PY{l+s+sd}{    коэффициента `h` (x \PYZlt{}\PYZhy{} x \PYZhy{} h * grad\PYZus{}f).}
\PY{l+s+sd}{    \PYZdq{}\PYZdq{}\PYZdq{}}
    \PY{n}{x} \PY{o}{=} \PY{n}{np}\PY{o}{.}\PY{n}{array}\PY{p}{(}\PY{n}{x0}\PY{p}{,} \PY{n}{dtype}\PY{o}{=}\PY{l+s+s1}{\PYZsq{}}\PY{l+s+s1}{float64}\PY{l+s+s1}{\PYZsq{}}\PY{p}{)}  \PY{c+c1}{\PYZsh{} текущее решение}
    \PY{n}{solutions} \PY{o}{=} \PY{p}{[}\PY{n}{x}\PY{o}{.}\PY{n}{copy}\PY{p}{(}\PY{p}{)}\PY{p}{]}             \PY{c+c1}{\PYZsh{} все решения}
    \PY{n}{grad} \PY{o}{=} \PY{n}{np}\PY{o}{.}\PY{n}{array}\PY{p}{(}\PY{n}{grad\PYZus{}f}\PY{p}{(}\PY{o}{*}\PY{n}{x}\PY{p}{)}\PY{p}{)}        \PY{c+c1}{\PYZsh{} градиент в текущей точке}
    \PY{n}{grad\PYZus{}norm} \PY{o}{=} \PY{n}{np}\PY{o}{.}\PY{n}{linalg}\PY{o}{.}\PY{n}{norm}\PY{p}{(}\PY{n}{grad}\PY{p}{)}
    \PY{n}{f\PYZus{}x} \PY{o}{=} \PY{n}{f}\PY{p}{(}\PY{o}{*}\PY{n}{x}\PY{p}{)}

    \PY{c+c1}{\PYZsh{} обновляем ответ, пока норма градиента не станет достаточно малой}
    \PY{k}{while} \PY{n}{grad\PYZus{}norm} \PY{o}{\PYZgt{}} \PY{n}{grad\PYZus{}min}\PY{p}{:}
        \PY{k}{if} \PY{n}{h\PYZus{}method} \PY{o}{==} \PY{l+s+s1}{\PYZsq{}}\PY{l+s+s1}{min}\PY{l+s+s1}{\PYZsq{}}\PY{p}{:}
            \PY{c+c1}{\PYZsh{} поиск h из условия минимума}
            \PY{n}{h} \PY{o}{=} \PY{l+m+mf}{1.0}
            \PY{k}{while} \PY{k+kc}{True}\PY{p}{:}
                \PY{n}{h} \PY{o}{=} \PY{n}{golden\PYZus{}section\PYZus{}search}\PY{p}{(}
                    \PY{k}{lambda} \PY{n}{h}\PY{p}{:} \PY{n}{f}\PY{p}{(}\PY{o}{*}\PY{p}{(}\PY{n}{x\PYZus{}i} \PY{o}{\PYZhy{}} \PY{n}{h} \PY{o}{*} \PY{n}{grad\PYZus{}x\PYZus{}i}
                                \PY{k}{for} \PY{n}{x\PYZus{}i}\PY{p}{,} \PY{n}{grad\PYZus{}x\PYZus{}i} \PY{o+ow}{in} \PY{n+nb}{zip}\PY{p}{(}\PY{n}{x}\PY{p}{,} \PY{n}{grad}\PY{p}{)}\PY{p}{)}\PY{p}{)}\PY{p}{,}
                    \PY{n}{a}\PY{o}{=}\PY{l+m+mi}{0}\PY{p}{,}
                    \PY{n}{b}\PY{o}{=}\PY{n}{h}
                \PY{p}{)}
                \PY{k}{if} \PY{n}{f}\PY{p}{(}\PY{o}{*}\PY{p}{(}\PY{n}{x} \PY{o}{\PYZhy{}} \PY{n}{h} \PY{o}{*} \PY{n}{grad}\PY{p}{)}\PY{p}{)} \PY{o}{\PYZlt{}} \PY{n}{f\PYZus{}x}\PY{p}{:}
                    \PY{k}{break}
        \PY{k}{elif} \PY{n+nb}{isinstance}\PY{p}{(}\PY{n}{h\PYZus{}method}\PY{p}{,} \PY{p}{(}\PY{n+nb}{int}\PY{p}{,} \PY{n+nb}{float}\PY{p}{)}\PY{p}{)}\PY{p}{:}
            \PY{c+c1}{\PYZsh{} постоянное значение h}
            \PY{n}{h} \PY{o}{=} \PY{n}{h\PYZus{}method}
        \PY{k}{else}\PY{p}{:}
            \PY{c+c1}{\PYZsh{} поиск h через двойное неравенство}
            \PY{k}{assert} \PY{n+nb}{len}\PY{p}{(}\PY{n}{h\PYZus{}method}\PY{p}{)} \PY{o}{==} \PY{l+m+mi}{2}
            \PY{n}{alpha} \PY{o}{=} \PY{n}{h\PYZus{}method}\PY{p}{[}\PY{l+m+mi}{0}\PY{p}{]}
            \PY{n}{beta} \PY{o}{=} \PY{n}{h\PYZus{}method}\PY{p}{[}\PY{l+m+mi}{1}\PY{p}{]}
            \PY{n}{h\PYZus{}min}\PY{p}{,} \PY{n}{h\PYZus{}max} \PY{o}{=} \PY{l+m+mi}{0}\PY{p}{,} \PY{l+m+mi}{1}
            \PY{n}{h} \PY{o}{=} \PY{l+m+mi}{1}

            \PY{c+c1}{\PYZsh{} двоичный поиск}
            \PY{k}{while} \PY{k+kc}{True}\PY{p}{:}
                \PY{n}{decrease} \PY{o}{=} \PY{n}{f\PYZus{}x} \PY{o}{\PYZhy{}} \PY{n}{f}\PY{p}{(}\PY{o}{*}\PY{p}{(}\PY{n}{x} \PY{o}{\PYZhy{}} \PY{n}{grad} \PY{o}{*} \PY{n}{h}\PY{p}{)}\PY{p}{)}
                \PY{n}{dot\PYZus{}product} \PY{o}{=} \PY{n}{grad\PYZus{}norm}\PY{o}{*}\PY{o}{*}\PY{l+m+mi}{2} \PY{o}{*} \PY{n}{h}
                \PY{k}{if} \PY{n}{alpha} \PY{o}{*} \PY{n}{dot\PYZus{}product} \PY{o}{\PYZgt{}} \PY{n}{decrease}\PY{p}{:}
                    \PY{n}{h\PYZus{}max} \PY{o}{=} \PY{n}{h}
                \PY{k}{elif} \PY{n}{beta} \PY{o}{*} \PY{n}{dot\PYZus{}product} \PY{o}{\PYZlt{}} \PY{n}{decrease}\PY{p}{:}
                    \PY{n}{h\PYZus{}min} \PY{o}{=} \PY{n}{h}
                \PY{k}{else}\PY{p}{:}
                    \PY{k}{break}
                \PY{n}{h} \PY{o}{=} \PY{n}{h\PYZus{}min} \PY{o}{+} \PY{p}{(}\PY{n}{h\PYZus{}max} \PY{o}{\PYZhy{}} \PY{n}{h\PYZus{}min}\PY{p}{)} \PY{o}{/} \PY{l+m+mi}{2}

        \PY{n}{x} \PY{o}{\PYZhy{}}\PY{o}{=} \PY{n}{h} \PY{o}{*} \PY{n}{grad}
        \PY{n}{f\PYZus{}x} \PY{o}{=} \PY{n}{f}\PY{p}{(}\PY{o}{*}\PY{n}{x}\PY{p}{)}
        \PY{n}{grad} \PY{o}{=} \PY{n}{np}\PY{o}{.}\PY{n}{array}\PY{p}{(}\PY{n}{grad\PYZus{}f}\PY{p}{(}\PY{o}{*}\PY{n}{x}\PY{p}{)}\PY{p}{)}
        \PY{n}{grad\PYZus{}norm} \PY{o}{=} \PY{n}{np}\PY{o}{.}\PY{n}{linalg}\PY{o}{.}\PY{n}{norm}\PY{p}{(}\PY{n}{grad}\PY{p}{)}
        \PY{n}{solutions}\PY{o}{.}\PY{n}{append}\PY{p}{(}\PY{n}{x}\PY{o}{.}\PY{n}{copy}\PY{p}{(}\PY{p}{)}\PY{p}{)}

    \PY{k}{if} \PY{n}{return\PYZus{}solutions}\PY{p}{:}
        \PY{k}{return} \PY{n}{solutions}
    \PY{k}{return} \PY{n}{x}
\end{Verbatim}
\end{tcolorbox}

    Минимум обеих функций лежит в точке \(x_i = 1, i \in [1, n]\). Проверим,
находит ли его реализованный метод.

    \begin{tcolorbox}[breakable, size=fbox, boxrule=1pt, pad at break*=1mm,colback=cellbackground, colframe=cellborder]
\prompt{In}{incolor}{6}{\boxspacing}
\begin{Verbatim}[commandchars=\\\{\}]
\PY{n}{x} \PY{o}{=} \PY{n}{gradient\PYZus{}descent}\PY{p}{(}\PY{n}{quadratic}\PY{p}{,} \PY{n}{quadratic\PYZus{}derivatives}\PY{p}{,} \PY{p}{(}\PY{l+m+mi}{0}\PY{p}{,} \PY{l+m+mi}{0}\PY{p}{)}\PY{p}{)}
\PY{n+nb}{print}\PY{p}{(}\PY{l+s+sa}{f}\PY{l+s+s1}{\PYZsq{}}\PY{l+s+s1}{Квадратичная функция / n = 2:}\PY{l+s+s1}{\PYZsq{}}\PY{p}{)}
\PY{n+nb}{print}\PY{p}{(}\PY{l+s+sa}{f}\PY{l+s+s1}{\PYZsq{}}\PY{l+s+s1}{x = }\PY{l+s+si}{\PYZob{}}\PY{n}{x}\PY{l+s+si}{\PYZcb{}}\PY{l+s+se}{\PYZbs{}n}\PY{l+s+s1}{\PYZsq{}}\PY{p}{)}

\PY{n}{x} \PY{o}{=} \PY{n}{gradient\PYZus{}descent}\PY{p}{(}\PY{n}{quadratic}\PY{p}{,} \PY{n}{quadratic\PYZus{}derivatives}\PY{p}{,} \PY{p}{(}\PY{l+m+mi}{0}\PY{p}{,} \PY{l+m+mi}{0}\PY{p}{,} \PY{l+m+mi}{0}\PY{p}{)}\PY{p}{)}
\PY{n+nb}{print}\PY{p}{(}\PY{l+s+sa}{f}\PY{l+s+s1}{\PYZsq{}}\PY{l+s+s1}{Квадратичная функция / n = 3:}\PY{l+s+s1}{\PYZsq{}}\PY{p}{)}
\PY{n+nb}{print}\PY{p}{(}\PY{l+s+sa}{f}\PY{l+s+s1}{\PYZsq{}}\PY{l+s+s1}{x = }\PY{l+s+si}{\PYZob{}}\PY{n}{x}\PY{l+s+si}{\PYZcb{}}\PY{l+s+se}{\PYZbs{}n}\PY{l+s+s1}{\PYZsq{}}\PY{p}{)}

\PY{n}{x} \PY{o}{=} \PY{n}{gradient\PYZus{}descent}\PY{p}{(}\PY{n}{quadratic}\PY{p}{,} \PY{n}{quadratic\PYZus{}derivatives}\PY{p}{,} \PY{p}{(}\PY{l+m+mi}{0}\PY{p}{,}\PY{p}{)} \PY{o}{*} \PY{l+m+mi}{5}\PY{p}{)}
\PY{n+nb}{print}\PY{p}{(}\PY{l+s+sa}{f}\PY{l+s+s1}{\PYZsq{}}\PY{l+s+s1}{Квадратичная функция / n = 5:}\PY{l+s+s1}{\PYZsq{}}\PY{p}{)}
\PY{n+nb}{print}\PY{p}{(}\PY{l+s+sa}{f}\PY{l+s+s1}{\PYZsq{}}\PY{l+s+s1}{x = }\PY{l+s+si}{\PYZob{}}\PY{n}{x}\PY{l+s+si}{\PYZcb{}}\PY{l+s+se}{\PYZbs{}n}\PY{l+s+s1}{\PYZsq{}}\PY{p}{)}

\PY{n}{x} \PY{o}{=} \PY{n}{gradient\PYZus{}descent}\PY{p}{(}\PY{n}{quadratic}\PY{p}{,} \PY{n}{quadratic\PYZus{}derivatives}\PY{p}{,} \PY{p}{(}\PY{l+m+mi}{0}\PY{p}{,}\PY{p}{)} \PY{o}{*} \PY{l+m+mi}{10}\PY{p}{)}
\PY{n+nb}{print}\PY{p}{(}\PY{l+s+sa}{f}\PY{l+s+s1}{\PYZsq{}}\PY{l+s+s1}{Квадратичная функция / n = 10:}\PY{l+s+s1}{\PYZsq{}}\PY{p}{)}
\PY{n+nb}{print}\PY{p}{(}\PY{l+s+sa}{f}\PY{l+s+s1}{\PYZsq{}}\PY{l+s+s1}{x = }\PY{l+s+si}{\PYZob{}}\PY{n}{x}\PY{l+s+si}{\PYZcb{}}\PY{l+s+se}{\PYZbs{}n}\PY{l+s+s1}{\PYZsq{}}\PY{p}{)}

\PY{n}{x} \PY{o}{=} \PY{n}{gradient\PYZus{}descent}\PY{p}{(}\PY{n}{rosenbrock}\PY{p}{,} \PY{n}{rosenbrock\PYZus{}derivatives}\PY{p}{,} \PY{p}{(}\PY{l+m+mi}{0}\PY{p}{,} \PY{l+m+mi}{0}\PY{p}{)}\PY{p}{)}
\PY{n+nb}{print}\PY{p}{(}\PY{l+s+s1}{\PYZsq{}}\PY{l+s+s1}{Функция Розенброка:}\PY{l+s+s1}{\PYZsq{}}\PY{p}{)}
\PY{n+nb}{print}\PY{p}{(}\PY{l+s+sa}{f}\PY{l+s+s1}{\PYZsq{}}\PY{l+s+s1}{x = }\PY{l+s+si}{\PYZob{}}\PY{n}{x}\PY{l+s+si}{\PYZcb{}}\PY{l+s+se}{\PYZbs{}n}\PY{l+s+s1}{\PYZsq{}}\PY{p}{)}
\end{Verbatim}
\end{tcolorbox}

    \begin{Verbatim}[commandchars=\\\{\}]
Квадратичная функция / n = 2:
x = [0.99999001 0.99999002]

Квадратичная функция / n = 3:
x = [0.99998807 0.9999882  1.        ]

Квадратичная функция / n = 5:
x = [0.99998746 0.99998765 1.         1.         1.        ]

Квадратичная функция / n = 10:
x = [0.99998714 0.99998738 1.         1.         1.         1.
 1.         1.         1.         1.        ]

Функция Розенброка:
x = [0.99995974 0.99991899]

    \end{Verbatim}

\newpage
    Теперь посмотрим, сколько итераций выполняют различные реализации
метода. Постоянное значение \(h\) подобрано вручную, оно близко к
максимальному, при котором достигается сходимость.

    \begin{tcolorbox}[breakable, size=fbox, boxrule=1pt, pad at break*=1mm,colback=cellbackground, colframe=cellborder]
\prompt{In}{incolor}{7}{\boxspacing}
\begin{Verbatim}[commandchars=\\\{\}]
\PY{n}{x0} \PY{o}{=} \PY{p}{(}\PY{l+m+mi}{0}\PY{p}{,} \PY{l+m+mi}{0}\PY{p}{)}
\PY{n}{sol} \PY{o}{=} \PY{n}{gradient\PYZus{}descent}\PY{p}{(}\PY{n}{quadratic}\PY{p}{,} \PY{n}{quadratic\PYZus{}derivatives}\PY{p}{,} \PY{n}{x0}\PY{p}{,}
                       \PY{n}{return\PYZus{}solutions}\PY{o}{=}\PY{k+kc}{True}\PY{p}{)}
\PY{n+nb}{print}\PY{p}{(}\PY{l+s+s1}{\PYZsq{}}\PY{l+s+s1}{Квадратичная функция / поиск минимума:}\PY{l+s+s1}{\PYZsq{}}\PY{p}{)}
\PY{n+nb}{print}\PY{p}{(}\PY{l+s+sa}{f}\PY{l+s+s1}{\PYZsq{}}\PY{l+s+s1}{x = }\PY{l+s+si}{\PYZob{}}\PY{n}{sol}\PY{p}{[}\PY{o}{\PYZhy{}}\PY{l+m+mi}{1}\PY{p}{]}\PY{l+s+si}{\PYZcb{}}\PY{l+s+s1}{, число итераций: }\PY{l+s+si}{\PYZob{}}\PY{n+nb}{len}\PY{p}{(}\PY{n}{sol}\PY{p}{)}\PY{l+s+si}{\PYZcb{}}\PY{l+s+se}{\PYZbs{}n}\PY{l+s+s1}{\PYZsq{}}\PY{p}{)}

\PY{n}{sol} \PY{o}{=} \PY{n}{gradient\PYZus{}descent}\PY{p}{(}\PY{n}{quadratic}\PY{p}{,} \PY{n}{quadratic\PYZus{}derivatives}\PY{p}{,} \PY{n}{x0}\PY{p}{,}
                       \PY{n}{h\PYZus{}method}\PY{o}{=}\PY{l+m+mf}{2e\PYZhy{}3}\PY{p}{,} \PY{n}{return\PYZus{}solutions}\PY{o}{=}\PY{k+kc}{True}\PY{p}{)}
\PY{n+nb}{print}\PY{p}{(}\PY{l+s+s1}{\PYZsq{}}\PY{l+s+s1}{Квадратичная функция / h = const:}\PY{l+s+s1}{\PYZsq{}}\PY{p}{)}
\PY{n+nb}{print}\PY{p}{(}\PY{l+s+sa}{f}\PY{l+s+s1}{\PYZsq{}}\PY{l+s+s1}{x = }\PY{l+s+si}{\PYZob{}}\PY{n}{sol}\PY{p}{[}\PY{o}{\PYZhy{}}\PY{l+m+mi}{1}\PY{p}{]}\PY{l+s+si}{\PYZcb{}}\PY{l+s+s1}{, число итераций: }\PY{l+s+si}{\PYZob{}}\PY{n+nb}{len}\PY{p}{(}\PY{n}{sol}\PY{p}{)}\PY{l+s+si}{\PYZcb{}}\PY{l+s+se}{\PYZbs{}n}\PY{l+s+s1}{\PYZsq{}}\PY{p}{)}

\PY{n}{sol} \PY{o}{=} \PY{n}{gradient\PYZus{}descent}\PY{p}{(}\PY{n}{quadratic}\PY{p}{,} \PY{n}{quadratic\PYZus{}derivatives}\PY{p}{,} \PY{n}{x0}\PY{p}{,}
                       \PY{n}{h\PYZus{}method}\PY{o}{=}\PY{p}{(}\PY{l+m+mf}{0.1}\PY{p}{,} \PY{l+m+mf}{0.9}\PY{p}{)}\PY{p}{,} \PY{n}{return\PYZus{}solutions}\PY{o}{=}\PY{k+kc}{True}\PY{p}{)}
\PY{n+nb}{print}\PY{p}{(}\PY{l+s+s1}{\PYZsq{}}\PY{l+s+s1}{Квадратичная функция / двойное неравенство:}\PY{l+s+s1}{\PYZsq{}}\PY{p}{)}
\PY{n+nb}{print}\PY{p}{(}\PY{l+s+sa}{f}\PY{l+s+s1}{\PYZsq{}}\PY{l+s+s1}{x = }\PY{l+s+si}{\PYZob{}}\PY{n}{sol}\PY{p}{[}\PY{o}{\PYZhy{}}\PY{l+m+mi}{1}\PY{p}{]}\PY{l+s+si}{\PYZcb{}}\PY{l+s+s1}{, число итераций: }\PY{l+s+si}{\PYZob{}}\PY{n+nb}{len}\PY{p}{(}\PY{n}{sol}\PY{p}{)}\PY{l+s+si}{\PYZcb{}}\PY{l+s+se}{\PYZbs{}n}\PY{l+s+s1}{\PYZsq{}}\PY{p}{)}

\PY{n}{sol} \PY{o}{=} \PY{n}{gradient\PYZus{}descent}\PY{p}{(}\PY{n}{rosenbrock}\PY{p}{,} \PY{n}{rosenbrock\PYZus{}derivatives}\PY{p}{,} \PY{n}{x0}\PY{p}{,}
                       \PY{n}{return\PYZus{}solutions}\PY{o}{=}\PY{k+kc}{True}\PY{p}{)}
\PY{n+nb}{print}\PY{p}{(}\PY{l+s+s1}{\PYZsq{}}\PY{l+s+s1}{Функция Розенброка / поиск минимума:}\PY{l+s+s1}{\PYZsq{}}\PY{p}{)}
\PY{n+nb}{print}\PY{p}{(}\PY{l+s+sa}{f}\PY{l+s+s1}{\PYZsq{}}\PY{l+s+s1}{x = }\PY{l+s+si}{\PYZob{}}\PY{n}{sol}\PY{p}{[}\PY{o}{\PYZhy{}}\PY{l+m+mi}{1}\PY{p}{]}\PY{l+s+si}{\PYZcb{}}\PY{l+s+s1}{, число итераций: }\PY{l+s+si}{\PYZob{}}\PY{n+nb}{len}\PY{p}{(}\PY{n}{sol}\PY{p}{)}\PY{l+s+si}{\PYZcb{}}\PY{l+s+se}{\PYZbs{}n}\PY{l+s+s1}{\PYZsq{}}\PY{p}{)}

\PY{n}{sol} \PY{o}{=} \PY{n}{gradient\PYZus{}descent}\PY{p}{(}\PY{n}{rosenbrock}\PY{p}{,} \PY{n}{rosenbrock\PYZus{}derivatives}\PY{p}{,} \PY{n}{x0}\PY{p}{,}
                       \PY{n}{h\PYZus{}method}\PY{o}{=}\PY{l+m+mf}{2e\PYZhy{}3}\PY{p}{,} \PY{n}{return\PYZus{}solutions}\PY{o}{=}\PY{k+kc}{True}\PY{p}{)}
\PY{n+nb}{print}\PY{p}{(}\PY{l+s+s1}{\PYZsq{}}\PY{l+s+s1}{Функция Розенброка / h = const:}\PY{l+s+s1}{\PYZsq{}}\PY{p}{)}
\PY{n+nb}{print}\PY{p}{(}\PY{l+s+sa}{f}\PY{l+s+s1}{\PYZsq{}}\PY{l+s+s1}{x = }\PY{l+s+si}{\PYZob{}}\PY{n}{sol}\PY{p}{[}\PY{o}{\PYZhy{}}\PY{l+m+mi}{1}\PY{p}{]}\PY{l+s+si}{\PYZcb{}}\PY{l+s+s1}{, число итераций: }\PY{l+s+si}{\PYZob{}}\PY{n+nb}{len}\PY{p}{(}\PY{n}{sol}\PY{p}{)}\PY{l+s+si}{\PYZcb{}}\PY{l+s+se}{\PYZbs{}n}\PY{l+s+s1}{\PYZsq{}}\PY{p}{)}

\PY{n}{sol} \PY{o}{=} \PY{n}{gradient\PYZus{}descent}\PY{p}{(}\PY{n}{rosenbrock}\PY{p}{,} \PY{n}{rosenbrock\PYZus{}derivatives}\PY{p}{,} \PY{n}{x0}\PY{p}{,}
                       \PY{n}{h\PYZus{}method}\PY{o}{=}\PY{p}{(}\PY{l+m+mf}{0.1}\PY{p}{,} \PY{l+m+mf}{0.9}\PY{p}{)}\PY{p}{,} \PY{n}{return\PYZus{}solutions}\PY{o}{=}\PY{k+kc}{True}\PY{p}{)}
\PY{n+nb}{print}\PY{p}{(}\PY{l+s+s1}{\PYZsq{}}\PY{l+s+s1}{Функция Розенброка / двойное неравенство:}\PY{l+s+s1}{\PYZsq{}}\PY{p}{)}
\PY{n+nb}{print}\PY{p}{(}\PY{l+s+sa}{f}\PY{l+s+s1}{\PYZsq{}}\PY{l+s+s1}{x = }\PY{l+s+si}{\PYZob{}}\PY{n}{sol}\PY{p}{[}\PY{o}{\PYZhy{}}\PY{l+m+mi}{1}\PY{p}{]}\PY{l+s+si}{\PYZcb{}}\PY{l+s+s1}{, число итераций: }\PY{l+s+si}{\PYZob{}}\PY{n+nb}{len}\PY{p}{(}\PY{n}{sol}\PY{p}{)}\PY{l+s+si}{\PYZcb{}}\PY{l+s+se}{\PYZbs{}n}\PY{l+s+s1}{\PYZsq{}}\PY{p}{)}
\end{Verbatim}
\end{tcolorbox}

    \begin{Verbatim}[commandchars=\\\{\}]
Квадратичная функция / поиск минимума:
x = [0.99999001 0.99999002], число итераций: 473

Квадратичная функция / h = const:
x = [0.99998552 0.99998587], число итераций: 1125

Квадратичная функция / двойное неравенство:
x = [0.99998917 0.99998926], число итераций: 421

Функция Розенброка / поиск минимума:
x = [0.99995974 0.99991899], число итераций: 1856

Функция Розенброка / h = const:
x = [0.99995531 0.99991017], число итераций: 4422

Функция Розенброка / двойное неравенство:
x = [0.99995858 0.99991667], число итераций: 531

    \end{Verbatim}

    \begin{tcolorbox}[breakable, size=fbox, boxrule=1pt, pad at break*=1mm,colback=cellbackground, colframe=cellborder]
\prompt{In}{incolor}{8}{\boxspacing}
\begin{Verbatim}[commandchars=\\\{\}]
\PY{n}{mesh} \PY{o}{=} \PY{n}{np}\PY{o}{.}\PY{n}{meshgrid}\PY{p}{(}\PY{n}{np}\PY{o}{.}\PY{n}{linspace}\PY{p}{(}\PY{o}{\PYZhy{}}\PY{l+m+mf}{0.5}\PY{p}{,} \PY{l+m+mi}{2}\PY{p}{)}\PY{p}{,} \PY{n}{np}\PY{o}{.}\PY{n}{linspace}\PY{p}{(}\PY{o}{\PYZhy{}}\PY{l+m+mi}{1}\PY{p}{,} \PY{l+m+mi}{2}\PY{p}{)}\PY{p}{)}
\PY{n}{fig}\PY{p}{,} \PY{p}{[}\PY{n}{ax1}\PY{p}{,} \PY{n}{ax2}\PY{p}{]} \PY{o}{=} \PY{n}{plt}\PY{o}{.}\PY{n}{subplots}\PY{p}{(}\PY{n}{ncols}\PY{o}{=}\PY{l+m+mi}{2}\PY{p}{)}
\PY{n}{fig}\PY{o}{.}\PY{n}{set\PYZus{}size\PYZus{}inches}\PY{p}{(}\PY{l+m+mi}{9}\PY{p}{,} \PY{l+m+mf}{3.5}\PY{p}{)}

\PY{n}{sol} \PY{o}{=} \PY{n}{gradient\PYZus{}descent}\PY{p}{(}\PY{n}{quadratic}\PY{p}{,} \PY{n}{quadratic\PYZus{}derivatives}\PY{p}{,} \PY{n}{x0}\PY{p}{,}
                       \PY{n}{h\PYZus{}method}\PY{o}{=}\PY{p}{(}\PY{l+m+mf}{0.1}\PY{p}{,} \PY{l+m+mf}{0.9}\PY{p}{)}\PY{p}{,} \PY{n}{return\PYZus{}solutions}\PY{o}{=}\PY{k+kc}{True}\PY{p}{)}
\PY{n}{cn} \PY{o}{=} \PY{n}{ax1}\PY{o}{.}\PY{n}{contourf}\PY{p}{(}\PY{o}{*}\PY{n}{mesh}\PY{p}{,} \PY{n}{quadratic}\PY{p}{(}\PY{o}{*}\PY{n}{mesh}\PY{p}{)}\PY{p}{,} \PY{n}{levels}\PY{o}{=}\PY{l+m+mi}{20}\PY{p}{,} \PY{n}{cmap}\PY{o}{=}\PY{n}{plt}\PY{o}{.}\PY{n}{cm}\PY{o}{.}\PY{n}{coolwarm}\PY{p}{)}
\PY{n}{sol} \PY{o}{=} \PY{n}{np}\PY{o}{.}\PY{n}{array}\PY{p}{(}\PY{n}{sol}\PY{p}{)}
\PY{n}{ax1}\PY{o}{.}\PY{n}{plot}\PY{p}{(}\PY{n}{sol}\PY{p}{[}\PY{p}{:}\PY{p}{,} \PY{l+m+mi}{0}\PY{p}{]}\PY{p}{,} \PY{n}{sol}\PY{p}{[}\PY{p}{:}\PY{p}{,} \PY{l+m+mi}{1}\PY{p}{]}\PY{p}{,} \PY{l+s+s1}{\PYZsq{}}\PY{l+s+s1}{k\PYZhy{}}\PY{l+s+s1}{\PYZsq{}}\PY{p}{,} \PY{n}{lw}\PY{o}{=}\PY{l+m+mf}{0.5}\PY{p}{)}
\PY{n}{fig}\PY{o}{.}\PY{n}{colorbar}\PY{p}{(}\PY{n}{cn}\PY{p}{,} \PY{n}{ax}\PY{o}{=}\PY{n}{ax1}\PY{p}{)}
\PY{n}{ax1}\PY{o}{.}\PY{n}{set\PYZus{}title}\PY{p}{(}\PY{l+s+s1}{\PYZsq{}}\PY{l+s+s1}{Квадратичная функция}\PY{l+s+s1}{\PYZsq{}}\PY{p}{)}

\PY{n}{sol} \PY{o}{=} \PY{n}{gradient\PYZus{}descent}\PY{p}{(}\PY{n}{rosenbrock}\PY{p}{,} \PY{n}{rosenbrock\PYZus{}derivatives}\PY{p}{,} \PY{n}{x0}\PY{p}{,}
                       \PY{n}{h\PYZus{}method}\PY{o}{=}\PY{p}{(}\PY{l+m+mf}{0.1}\PY{p}{,} \PY{l+m+mf}{0.9}\PY{p}{)}\PY{p}{,} \PY{n}{return\PYZus{}solutions}\PY{o}{=}\PY{k+kc}{True}\PY{p}{)}
\PY{n}{cn} \PY{o}{=} \PY{n}{ax2}\PY{o}{.}\PY{n}{contourf}\PY{p}{(}\PY{o}{*}\PY{n}{mesh}\PY{p}{,} \PY{n}{rosenbrock}\PY{p}{(}\PY{o}{*}\PY{n}{mesh}\PY{p}{)}\PY{p}{,} \PY{n}{levels}\PY{o}{=}\PY{l+m+mi}{20}\PY{p}{,} \PY{n}{cmap}\PY{o}{=}\PY{n}{plt}\PY{o}{.}\PY{n}{cm}\PY{o}{.}\PY{n}{coolwarm}\PY{p}{)}
\PY{n}{sol} \PY{o}{=} \PY{n}{np}\PY{o}{.}\PY{n}{array}\PY{p}{(}\PY{n}{sol}\PY{p}{)}
\PY{n}{ax2}\PY{o}{.}\PY{n}{plot}\PY{p}{(}\PY{n}{sol}\PY{p}{[}\PY{p}{:}\PY{p}{,} \PY{l+m+mi}{0}\PY{p}{]}\PY{p}{,} \PY{n}{sol}\PY{p}{[}\PY{p}{:}\PY{p}{,} \PY{l+m+mi}{1}\PY{p}{]}\PY{p}{,} \PY{l+s+s1}{\PYZsq{}}\PY{l+s+s1}{k\PYZhy{}}\PY{l+s+s1}{\PYZsq{}}\PY{p}{,} \PY{n}{lw}\PY{o}{=}\PY{l+m+mf}{0.5}\PY{p}{)}
\PY{n}{fig}\PY{o}{.}\PY{n}{colorbar}\PY{p}{(}\PY{n}{cn}\PY{p}{,} \PY{n}{ax}\PY{o}{=}\PY{n}{ax2}\PY{p}{)}
\PY{n}{ax2}\PY{o}{.}\PY{n}{set\PYZus{}title}\PY{p}{(}\PY{l+s+s1}{\PYZsq{}}\PY{l+s+s1}{Функция Розенброка}\PY{l+s+s1}{\PYZsq{}}\PY{p}{)}
\end{Verbatim}
\end{tcolorbox}

            \begin{tcolorbox}[breakable, size=fbox, boxrule=.5pt, pad at break*=1mm, opacityfill=0]
\prompt{Out}{outcolor}{8}{\boxspacing}
\begin{Verbatim}[commandchars=\\\{\}]
Text(0.5, 1.0, 'Функция Розенброка')
\end{Verbatim}
\end{tcolorbox}
        
    \begin{center}
    \adjustimage{max size={0.9\linewidth}{0.9\paperheight}}{output_12_1.png}
    \end{center}
    { \hspace*{\fill} \\}
    

    % Add a bibliography block to the postdoc
    
    
    
\end{document}
